%%%%%%%%%%%%%%%%%%%%%%%%%%%%%%%%%%%%%%%%%%%%%%%%%%%%%%%%%%%%
%%% LIVECOMS ARTICLE TEMPLATE FOR BEST PRACTICES GUIDE
%%% ADAPTED FROM ELIFE ARTICLE TEMPLATE (8/10/2017)
%%%%%%%%%%%%%%%%%%%%%%%%%%%%%%%%%%%%%%%%%%%%%%%%%%%%%%%%%%%%
%%% PREAMBLE
\documentclass[9pt,tutorial]{livecoms}
% Use the 'onehalfspacing' option for 1.5 line spacing
% Use the 'doublespacing' option for 2.0 line spacing
% Use the 'lineno' option for adding line numbers.
% Use the "ASAPversion' option following article acceptance to add the DOI and relevant dates to the document footer.
% Use the 'pubversion' option for adding the citation and publication information to the document footer, when the LiveCoMS issue is finalized.
% The 'bestpractices' option for indicates that this is a best practices guide.
% Omit the bestpractices option to remove the marking as a LiveCoMS paper.
% Please note that these options may affect formatting.

\usepackage{lipsum} % Required to insert dummy text
\usepackage[version=4]{mhchem}
\usepackage{siunitx}
\DeclareSIUnit\Molar{M}
\usepackage[italic]{mathastext}
\graphicspath{{figures/}}

%%%%%%%%%%%%%%%%%%%%%%%%%%%%%%%%%%%%%%%%%%%%%%%%%%%%%%%%%%%%
%%% IMPORTANT USER CONFIGURATION
%%%%%%%%%%%%%%%%%%%%%%%%%%%%%%%%%%%%%%%%%%%%%%%%%%%%%%%%%%%%

\newcommand{\versionnumber}{1.3}  % you should update the minor version number in preprints and major version number of submissions.
\newcommand{\githubrepository}{\url{https://github.com/myaccount/homegithubrepository}}  %this should be the main github repository for this article

%%%%%%%%%%%%%%%%%%%%%%%%%%%%%%%%%%%%%%%%%%%%%%%%%%%%%%%%%%%%
%%% ARTICLE SETUP
%%%%%%%%%%%%%%%%%%%%%%%%%%%%%%%%%%%%%%%%%%%%%%%%%%%%%%%%%%%%
\title{This is the title [Article v\versionnumber]}

\author[1*]{Firstname Middlename Surname}
\author[1,2\authfn{1}\authfn{3}]{Firstname Middlename Familyname}
\author[2\authfn{1}\authfn{4}]{Firstname Initials Surname}
\author[2*]{Firstname Surname}
\affil[1]{Institution 1}
\affil[2]{Institution 2}

\corr{email1@example.com}{FMS}  % Correspondence emails.  FMS and FS are the appropriate authors initials.
\corr{email2@example.com}{FS}

\orcid{Author 1 name}{AAAA-BBBB-CCCC-DDDD}
\orcid{Author 2 name}{EEEE-FFFF-GGGG-HHHH}

\contrib[\authfn{1}]{These authors contributed equally to this work}
\contrib[\authfn{2}]{These authors also contributed equally to this work}

\presentadd[\authfn{3}]{Department, Institute, Country}
\presentadd[\authfn{4}]{Department, Institute, Country}

\blurb{This LiveCoMS document is maintained online on GitHub at \githubrepository; to provide feedback, suggestions, or help improve it, please visit the GitHub repository and participate via the issue tracker.}

%%%%%%%%%%%%%%%%%%%%%%%%%%%%%%%%%%%%%%%%%%%%%%%%%%%%%%%%%%%%
%%% PUBLICATION INFORMATION
%%% Fill out these parameters when available
%%% These are used when the "pubversion" option is invoked
%%%%%%%%%%%%%%%%%%%%%%%%%%%%%%%%%%%%%%%%%%%%%%%%%%%%%%%%%%%%
\pubDOI{10.XXXX/YYYYYYY}
\pubvolume{<volume>}
\pubissue{<issue>}
\pubyear{<year>}
\articlenum{<number>}
\datereceived{Day Month Year}
\dateaccepted{Day Month Year}


%%% Shortcuts and macros
\newcommand{\dgsolv}{\Delta G_\textup{solv}}
\newcommand{\software}{\emph}
\newcommand{\todo}{\textcolor{red}}

\newenvironment{code}{\par \noindent \ttfamily}{}

%%%%%%%%%%%%%%%%%%%%%%%%%%%%%%%%%%%%%%%%%%%%%%%%%%%%%%%%%%%%
%%% ARTICLE START
%%%%%%%%%%%%%%%%%%%%%%%%%%%%%%%%%%%%%%%%%%%%%%%%%%%%%%%%%%%%

\begin{document}

\begin{frontmatter}
\maketitle

\begin{abstract}
Grid inhomogeneous solvation theory (GIST) is a method to compute the free energy of hydration of a compound on a 3-dimensional grid.
The high spatial resolution of the GIST output, as well as the decomposition into energy and entropy contributions, allow for highly detailed analyses on both proteins and small molecules. However, this versatility also comes with a higher entry barrier for new users.

In this tutorial, we aim to guide the reader through the most common steps involved in a GIST analysis at the example of the streptavidin-biotin complex.
Furthermore, we discuss the theory of GIST with a focus on practical aspects, and show several pitfalls and technical difficulties that may occur during a GIST study.
We assume familiarity with molecular dynamics (MD) simulations as well as the AmberTools package.

%This particular document provides a skeleton illustrating key sections for a Tutorial document.
%Please see the sample \texttt{sample-document.tex} in \url{github.com/livecomsjournal/article_templates/templates} for additional information on and examples of using the LiveCoMS LaTeX class.
%Here we also assume familiarity with LaTeX and knowledge of how to include figures, tables, etc.; if you want examples, see the sample just referenced.
%
%In your work, in this particular slot, please provide an abstract of no more than 250 words.
%Your abstract should explain the main contributions of your article, and should not contain any material that is not included in the main text.
%Please note that your abstract, plus the authorship material following it, must not extend beyond the title page or modifications to the LaTeX class will likely be needed.
\end{abstract}

\end{frontmatter}




\section{Introduction}

%Here you would explain what problem you are tackling and briefly motivate your work.
%
%In this particular template, we have removed most of the usage examples which occur in \texttt{sample-document.tex} to provide a minimal template you can modify; however, we retain a couple of examples illustrating more unusual features of our templates/article class, such as the checklists, and information on algorithms and pseudocode.
%
%Keep in mind, as you prepare your manuscript, that you should plan for a representative image  which will be used to highlight your article on the journal website and publications. Usually, this would be one of your figures, but it must also be uploaded separately upon article submission. We give specific guidelines for this image on the journal website in the section on article submission (see \url{https://livecomsjournal.github.io/authors/policies/index.html#article-submission}).
%
%Additionally, for well-formatted manuscripts, we recommend that you let LaTeX handle figure/table placement for you as much as possible, so please avoid specifying strenuous float instructions like `[h!]` and `[H]` as much as possible.

\subsection{Scope}
The aim of this tutorial is to provide an example of a GIST workflow that can be easily adapted towards different systems.
We present a GIST study of the streptavidin-biotin complex, which allows us to show many common aspects of GIST studies.
We present solvation analysis of a small molecule (biotin) as well as a biomolecule (streptavidin).
For both systems, we show how to interpret the three-dimensional contributions of solvation free energy in a binding pocket.
Furthermore, we compute the contribution of solvation free energy to binding, which requires accurate post-processing of the GIST output to avoid unfavorable summation of bias in the GIST calculation.
We therefore believe that this example study covers a wide range of applications of the GIST method.

Furthermore, we discuss several technical aspects of GIST analyses, such as the normalization of voxel values and how to deal with double counting of energy contributions. 
%Tutorials should endeavor to cover the specific task at hand, and also highlight how the steps might need to be modified (or additional care might need to be taken at particular points) to handle more general cases.
%
%The scope of the tutorial, as well as the expected proficiencies / outcomes for researchers who complete the tutorial, should be clearly defined.
%This will often happen in a specific section or subsection in the article itself.

\section{Prerequisites}

%Here you would identify prerequisites/background knowledge that are assumed by your work, as well as any software/license requirements.

\subsection{Background knowledge}
%Tutorials should clearly define what concepts or abilities researchers will need to complete the tutorial (e.g., some proficiency in Python; experience with Jupyter notebooks; knowledge of classical MD; etc).

\subsection{Software/system requirements}
%Tutorials should clearly define what system and/or software requirements the researcher will need to complete the tutorial (e.g., VMD version 1.9 or newer, AMBER, etc.). Tutorials requiring specific software packages must provide instructions and files for the referenced version of the software.



\section{Theory}
GIST is an implementation of Inhomogeneous Solvation Theory (IST) \cite{Lazaridis1998} that discretizes  the free energy of solvation $\dgsolv$ on a three-dimensional grid. 
It was first devised by Nguyen et al. \cite{Nguyen2012} to overcome the limitation of IST which was commonly used for regions of high water occupancy and its implementation in \software{cpptraj} was thoroughly described in \cite{Ramsey2016}.
Here, we only provide a short overview of the theory behind GIST.
For more detailed information, we recommend one of the more recent publications on developments in GIST. \cite{Kraml2020}\cite{Chen2021}


\subsection{Solvation Entropy}
The solvation entropy of a system is totality of the contribution of the  solute-water entropy and the water-water entropy. 

\begin{equation}
	\Delta S_\textit{solv} = \Delta S_\textit{sw} + \Delta S_\textit{ww}
\end{equation}
Here the solvation entropy is approximated from the contribution of solute-water interaction. 
\textbf{STILLNEEDS REVISON + ADD CITATIONS} The quantity $k_\textit{b}$represents the Boltzman constant, 
$\rho^\textit{o}$ is the bulk number density, $g_\textit{sw}\left(\textbf{r},\omega \right)$ is the solute-water 
pair correlation function in the solute's frame of reference, \textbf{r} is the coordinates of the oxygen atom of the water molecule,  
$\omega$ is the Euler angle, and the factor $\frac{1}{8\pi^2}$ is the normalization constant. In bulk and in regions where the orientational 
entropy is uniform, the quantity $g_\textit{sw}\left(\textbf{r},\omega \right)$ is unity which leads to the first-order solvation entropy of 
bulk to be zero. Also, this quantity reaches unity as the distance from the solute molecule increases which leads for the calculation of 
the solvation entropy to be an approximation of the integral around the solute atom. 

\begin{equation}
	\Delta S_\textit{solv} \approx \Delta S_\textit{sw} \equiv -k_\textit{B} \frac{\rho^\textit{0}}{8\pi^\textit{2}} \int g_\textit{sw} \left(\textbf{r}, \omega \right) d\textbf{r}d\omega
\end{equation}

The solvation entropy is further broken down in terms of the contribution from translational and orientational terms.
\begin{equation}
	\Delta S_\textit{solv} = \Delta S_\textit{trans} + \Delta S_\textit{orient}
\end{equation}

\subsection{Entropy Calculations in Cpptraj}
In \software{cpptraj} the calculation of  solvation entropy is handled into two methods.

The first method assumes that the position ($r$) is independent of the orientation ($\omega$). This approximation allows for the 
splitting of the contributions of the translational and the orientational entropy. The nearest neighbor (NN) approach is used to 
evaluate each expression where $N_\textit{k}$ is the number of water molecules found in voxel k, $\gamma$ is the Euler's constant 
that accounts for the bias in naive entropy estimator, and $g_\textit{NN}$ is the nearest neighbor estimate. In the nearest neighbor
estimate, a voxel is considered to be a neighboring voxel when it shares any vertices of the voxel of interest. 
\begin{equation}
	g_\textit{vox} \left( \textbf{r}, \mathbf{\omega} \right) \approx g_\textit{vox} \left( \textbf{r})g_\textit{vox}(\mathbf{\omega} \right)
\end{equation}

\begin{equation}
	S_{k}^\textit{trans} \approx \frac{-R}{N_\textit{k}} \left( \sum _{i=1}^{N_k} ln g_{NN, \textit{i}} \left( \textbf{r} \right) + \gamma \right)
\end{equation}


\begin{equation}
	S_{k}^\textit{orient} \approx \frac{-R}{N_\textit{k}} \left( \sum _{i=1}^{N_k} ln g_{NN, \textit{i}} \left( \mathbf {\omega} \right) + \gamma \right)
\end{equation}

The second method directly calculates the solvation entropy by evaluating the six-dimensional integral (3 for position and 3 for orientation) 
using the nearest-neighbor approach. Unlike the first method, the nearest neighbor estimate for the sixth-order entropy considers all the water 
found within the same and neighboring voxels.  
\begin{equation}
	S_\textit{k} \approx \frac{-R}{N_\textit{k}} \left( \sum _{i=1}^{N_k} ln g_{NN, \textit{i}} \left( \textbf{r}, \mathbf {\omega} \right) + \gamma \right)
\end{equation}


\subsection{Solvation Energy}
The solvation energy is readily calculated by summing up the contributions of the water-water interaction and the water-solute interaction.
\begin{equation}
	\Delta E_\textit{solv} = \Delta E_\textit{sw} + \Delta E_\textit{ww}
\end{equation}

\begin{equation}
	\Delta E_\textit{sw} \equiv \frac{1}{8\pi^2} \int g_\textit{sw}\left(\omega|\textbf{r}\right) U_\textit{sw}\left(\textbf{r}, \omega\right) d\omega
\end{equation}

Similar to the entropy integrals, the solute-water solvation energy integral also decays with increasing distance from the solute. Hence, the solvation energy can be approximated by local spatial integrals. However, there is a complication arising from this approximation. Water molecules outside the region of interest will also be accounted for in the pairwise water term. For example, given two different regions $R_a$ and $R_b$, will not sum up to the total energy of the two regions due to double counting. 

\begin{equation}
\begin{aligned}
	\Delta E_\textit{ww} \equiv & \left(\frac{1}{8\pi^2} \right)^2 \rho^o \int g_\textit{sw}\left(\omega|\textbf{r}\right) \\
	& \times \left[g_\textit{sw}\left(\textbf{r}^\prime , \omega^\prime \right) - g^o_\textit{ww} \left(\textbf{r}, \omega, \textbf{r}^\prime, \omega^\prime \right)\right] \\
	& \times U_\textit{ww}\left(\textbf{r}, \omega, \textbf{r}^\prime, \omega^\prime\right)d\omega d\textbf{r}^\prime d\omega^\prime
\end{aligned}
\end{equation}
There are two ways in which this double counting was accounted for in the original paper. First, using the water-displacement method. This method is applicable when a determining the energetic cost of kicking out a water molecule. 
\begin{equation}
	\Delta E^R_\textit{ww} = n^\textit{R} E^\textit{bulk}_\textit{ww} - E^\textit{R,corr}_\textit{ww}
\end{equation}

The second method uses normalized-water properties to account the difference in water properties of two different regions. The normalized water property is defined as the water property per water molecule. 
\begin{equation}
	\Delta E^\textit{R,norm}_\textit{ww} = E^\textit{R,norm}_\textit{ww} - 2E^\textit{bulk}_\textit{ww}
\end{equation}

\subsection{PME implementation}

Here, we aim to guide the reader through a GIST analysis of the streptavidin-biotin complex. The goal is to
\begin{enumerate}
	\item compute the binding free energy of the complex from the individual energy contributions combined with the contribution of hydration computed using GIST.
	\item visualize local contributions to $\dgsolv$ in the streptavidin binding pocket as well as around biotin.
\end{enumerate}

\subsection{Streptavidin/Biotin}
\subsection{Tutorial data}
We will use the 1STP crystal structure to start our calculations.
For the sake of simplicity, you can download prepared and solvated Amber topologies and structures based on 1STP by running
\begin{code}
git clone git@github.com:liedllab/gist-tutorial.git
\end{code}
This will download the tutorial files (including both the manuscript and the examples in the \software{code} folder).
\subsection{System Preparation and Equilibration}
When calculating solvation free energy differences between a complex and the corresponding monomers (the dissociated state), there are two options for dealing with configurational diversity in the structures.
\todo{Maybe Valentin knows a citation for that?}
One can choose to sample the complex and dissociated states individually, or to sample only one of them and assume that there is sufficient overlap in the sampled conformations.
%One can choose to sample only the complex, assuming that all relevant conformations of the dissociated state are also found in the complex.
%The probability distribution of the dissociated state can be obtained by reweighting the probabilities in the complex using the solvation free energies.
%Alternatively, one can sample the complex and dissociated states individually.
%This approach does not require an overlap between the conformational probabilities of the complex and dissociated states.
%However, it is prone to statistical errors due to uncertainties in the ensembles, and requires the free energy contributions from the internal degrees of freedom of the monomers (in both the complex and the monomer states) to be known. \todo{This paragraph does not fit here well. Maybe put it somewhere else.}

Here, we choose not to include any sampling of the conformational states. It is therefore important that the conformations of the monomers in the complex and dissociated states match exactly. We provide a script called \software{cpptraj\_remove\_mol.sh} that removes a single molecule from a topology and structure using \software{cpptraj} without altering the rest of the system. Use this to create monomer topologies of biotin and streptavidin.


\subsection{Running GIST}
\subsection{Visualizing $\dgsolv$}
\subsection{Contribution of Hydration to Binding}

%\section{Content and links}
%
%A tutorial will normally draw on additional files and materials; clearly indicate where and how these are available, with links, and how they are being archived for the long-term and maintained so they stay current.
%You will likely want to reference your GitHub repository as a central point to access all of this information, and then the GitHub repository may link out to other content as needed.

\section{Checklists}
%Tutorials do not necessarily require the use of a checklist as in Best Practices documents; however, they can include these if desired.
%Several useful checklist formats are available, with examples presented in \texttt{sample-document.tex} in \url{github.com/livecomsjournal/article_templates/templates}.
%One example is shown here.

% Here is a single-column checklist that consists of multiple sub-checklists
\begin{Checklists}
	
\begin{checklist}{Simulation settings}
\end{checklist}

\begin{checklist}{Obtaining absolute $\Delta G_\textup{solv}$ }
\end{checklist}

\begin{checklist}{A list}
\textbf{Single-column checklists are also straightforward by removing the asterisk}
\begin{itemize}
\item First thing let's do an item which breaks across lines to see how that looks
\item Also remember
\item And finally
\end{itemize}
\end{checklist}
%
%\begin{checklist}{Another list}
%\textbf{This is some further description.}
%\begin{itemize}
%\item First thing
%\item Also remember
%\item And finally
%\end{itemize}
%\end{checklist}

\end{Checklists}








\section*{Author Contributions}
%%%%%%%%%%%%%%%%
% This section mustt describe the actual contributions of
% author. Since this is an electronic-only journal, there is
% no length limit when you describe the authors' contributions,
% so we recommend describing what they actually did rather than
% simply categorizing them in a small number of
% predefined roles as might be done in other journals.
%
% See the policies ``Policies on Authorship'' section of https://livecoms.github.io
% for more information on deciding on authorship and author order.
%%%%%%%%%%%%%%%%

(Explain the contributions of the different authors here)

% We suggest you preserve this comment:
For a more detailed description of author contributions,
see the GitHub issue tracking and changelog at \githubrepository.

\section*{Other Contributions}
%%%%%%%%%%%%%%%
% You should include all people who have filed issues that were
% accepted into the paper, or that upon discussion altered what was in the paper.
% Multiple significant contributions might mean that the contributor
% should be moved to authorship at the discretion of the a
%
% See the policies ``Policies on Authorship'' section of https://livecoms.github.io for
% more information on deciding on authorship and author order.
%%%%%%%%%%%%%%%

(Explain the contributions of any non-author contributors here)
% We suggest you preserve this comment:
For a more detailed description of contributions from the community and others, see the GitHub issue tracking and changelog at \githubrepository.

\section*{Potentially Conflicting Interests}
%%%%%%%
%Declare any potentially competing interests, financial or otherwise
%%%%%%%

Declare any potentially conflicting interests here, whether or not they pose an actual conflict in your view.

\section*{Funding Information}
%%%%%%%
% Authors should acknowledge funding sources here. Reference specific grants.
%%%%%%%
FMS acknowledges the support of NSF grant CHE-1111111.

\section*{Author Information}
\makeorcid

\bibliography{bibliography}

%%%%%%%%%%%%%%%%%%%%%%%%%%%%%%%%%%%%%%%%%%%%%%%%%%%%%%%%%%%%
%%% APPENDICES
%%%%%%%%%%%%%%%%%%%%%%%%%%%%%%%%%%%%%%%%%%%%%%%%%%%%%%%%%%%%

%\appendix


\end{document}
